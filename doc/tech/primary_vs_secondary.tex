
\documentclass{article}

\usepackage{ComPASS_notations}

\begin{document}

We are solving for the non linear system of equations $\mathcal{R} (X_\D) = 0$.
For each control volume $\nu \in \CV$, $\mathcal{R_\nu}$ is made of
conservation equations $R_\nu \(X_\D\)=0$ that involve adjacent degrees of
freedom and local closure laws $L_\nu \(X_\D\)=0$ that only depends on $X_\nu$,
local meaning that $\nabla_X_\D$ is block diagonal.

Let us write the equation in vector form:

%
\begin{eqnarray}
\label{NonLinearSystem}
\mathbf{0} = \mathcal{R} (X_\D) := 
\left\{\begin{array}{ll}
\left(\begin{array}{c}
R_\nu (X_\D) \\
L_\nu (X_\D)
\end{array}\right) 
\ \nu \in \CV, \\
\end{array}\right.
\end{eqnarray}
%

Let us denote:
\begin{itemize}
\item $X^{\nat}=\(X^{\nat}_{\nu}\)$ the {\em physical / natural} unknowns 
\item $X^{\sol}=\(X^{\sol}_{\nu}\)$ the unkwowns used to solve the problem.
\end{itemize}

such that: $X^{\sol} = \cG \(X^{\nat}\)$
and $X^{\sol}_{\nu} = \cG_\nu \( X^{\nat}_{\nu} \)$


We assume that $X^{\sol}_{\nu}$ writes in vector form :
$$
X^{\sol}_{\nu}=\(
\begin{array}{cc}
X^{\primu}_{\nu} \\
X^{\secdu}_{\nu}
\end{array}
\)
$$
and is such that the square matrix 
$\nabla_{X^{\secdu}_{\nu}} L_\nu$ is non-singular
for all $\nu \in \CV$.

Most of the time $\cG$ is a mere permutation matrix.

When solving \eqref{NonLinearSystem} with respect to $X^{\sol}$ with a Newton
algorithm, we are seeking increments such that:

$$
\nabla_{X^{\sol}}\mathcal{R} \Delta_X = - \mathcal{R} \( X \)
$$

where the diagonal of the Jacobian $\nabla_{X^{\sol}}\mathcal{R}$ is made
of blocks of the form:

$$
\(\begin{array}{cc}
\nabla_{X^{\primu}_{\nu}} R_{\nu} &
\nabla_{X^{\secdu}_{\nu}} R_{\nu} \\
\nabla_{X^{\primu}_{\nu}} L_{\nu} &
\nabla_{X^{\secdu}_{\nu}} L_{\nu}
\end{array}\)
$$

As the rows involving $L^{\sol}_{\nu}$ are null except in this block (local
nature of the closure laws) we can eliminate these equations (which can be seen
as a Schur complement without fill-in):

Indeed:

$$
\nabla_{X^{\primu}_{\nu}} L_{\nu} \Delta_{X^{\primu}_{\nu}} +
\nabla_{X^{\secdu}_{\nu}} L_{\nu} \Delta_{X^{\secdu}_{\nu}}
= L_{\nu}
$$

gives:

$$
\Delta_{X^{\secdu}_{\nu}}
=  \(\nabla_{X^{\secdu}_{\nu}} L_{\nu} \)^{-1} \(
L^{\sol}_{\nu} - 
\nabla_{X^{\primu}_{\nu}} L_{\nu} \Delta_{X^{\primu}_{\nu}} 
\)
$$

At each step of the Newton algorithm we can choose 
$( X^{\sol}_{\nu} )_{\nu\in\CV}$ that 
$ L^{\sol}_{\nu} = 0 $ (projection onto the closure laws ? is this the role
of the {\em flash}?).

Then for each degree of freedom $\nu'$ the rows involving $R^{\sol}_{\nu}$
rewrite:
$$
\sum_{\nu'}
\(
\nabla_{X^{\primu}_{\nu'}} R_{\nu} \Delta_{X^{\primu}_{\nu'}}+
\nabla_{X^{\secdu}_{\nu'}} R_{\nu} \Delta_{X^{\secdu}_{\nu'}}
\)
= R_{\nu}
$$
 
$$
\sum_{\nu'}
\[
\nabla_{X^{\primu}_{\nu'}} R_{\nu} \Delta_{X^{\primu}_{\nu'}}+
\nabla_{X^{\secdu}_{\nu'}} R_{\nu} \(
\(\nabla_{X^{\secdu}_{\nu'}} L_{\nu'} \)^{-1} \(
L_{\nu'} - 
\nabla_{X^{\primu}_{\nu'}} L_{\nu'} \Delta_{X^{\primu}_{\nu'}} 
\)
\)
\]
= R_{\nu}
$$
 
\begin{eqnarray}
\label{eq:newtonjacobian}
\left.\begin{array}{lr}
\sum_{\nu'}
\[
\nabla_{X^{\primu}_{\nu'}} R_{\nu} 
- \nabla_{X^{\secdu}_{\nu'}} R_{\nu}
\( \nabla_{X^{\secdu}_{\nu'}} L_{\nu'} \)^{-1}
\nabla_{X^{\primu}_{\nu'}} L_{\nu'}
\] \Delta_{X^{\primu}_{\nu'}}
= \\
R_{\nu} - \sum_{\nu'} \[
\nabla_{X^{\secdu}_{\nu'}} R_{\nu} 
\(\nabla_{X^{\secdu}_{\nu'}} L_{\nu'} \)^{-1} L_{\nu'}
\]
\end{array}\right.
\end{eqnarray}
 
We also have:
$$
\nabla_{X^{\sol}_{\nu}} F
= \nabla_{X^{\nat}_{\nu}} F \nabla_{X^{\sol}_{\nu}} X^{\nat}_{\nu}
$$
and:
$$
\nabla_{X^{\sol}_{\nu}} X^{\nat}_{\nu}
= \nabla_{X^{\sol}_{\nu}} \( \cG_{\nu}^{-1}\)
= \( \nabla_{X^{\sol}_{\nu}} \cG_{\nu} \) ^{-1}
$$

Let us note : 
$$
\G_\nu := \nabla_{X^{\sol}_{\nu}} \cG_{\nu}
$$

And:
$$
\nabla_{X^{\sol}_{\nu}} F
= \nabla_{X^{\nat}_{\nu}} F {\G_{\nu}}^{-1} 
$$

with the notations:

$$
{\G_{\nu}}^{-1} = \[\,
\iGcol^{\primu}_{\nu} \, \iGcol^{\secdu}_{\nu} 
\,\]
$$
$$
J^{R}_{\nu,\nu'} = \nabla_{X^{\nat}_{\nu'}} R_{\nu} 
$$
$$
J^{L}_{\nu,\nu'} = \nabla_{X^{\nat}_{\nu'}} L_{\nu} 
$$

(note that when $\nu\neq\nu'$ we have $J^{L}_{\nu,\nu'}=0$).

Finally, \eqref{eq:newtonjacobian} becomes:

\begin{eqnarray}
\label{eq:newtonjacobian2}
\left.\begin{array}{lr}
\sum_{\nu'}
\[
 J^{R}_{\nu,\nu'}
-  J^{R}_{\nu,\nu'} \iGcol^{\secdu}_{\nu'}
\(  J^{L}_{\nu',\nu'} \iGcol^{\secdu}_{\nu'} \)^{-1} 
 J^{L}_{\nu',\nu'}
\] \iGcol^{\primu}_{\nu'} \Delta_{X^{\primu}_{\nu'}}
\\ =
R_{\nu} - \sum_{\nu'} \[
J^{R}_{\nu,\nu'} \iGcol^{\secdu}_{\nu'} 
\( J^{L}_{\nu',\nu'} \iGcol^{\secdu}_{\nu'} \)^{-1}
L_{\nu'} 
\]
\end{array}\right.
\end{eqnarray}

\begin{rmk}
When $\cG$ is a permutation matrix the effect of  the $\iGcol^{\cdot}_{\nu}$
terms is just to extract columns/rows and is implemented manually in the code.
\end{rmk}

\begin{rmk}
The terms $A_{\nu,\nu'}=J^{R}_{\nu,\nu'} \iGcol^{\secdu}_{\nu'} 
\( J^{L}_{\nu',\nu'} \iGcol^{\secdu}_{\nu'} \)^{-1}$ may be stored as they 
appear twice, so that \eqref{eq:newtonjacobian2} would become:
\begin{eqnarray}
\sum_{\nu'}
\[
 J^{R}_{\nu,\nu'}
-  A_{\nu,\nu'} 
 J^{L}_{\nu',\nu'} 
\] \iGcol^{\primu}_{\nu'} \Delta_{X^{\primu}_{\nu'}}
=
R_{\nu} - \sum_{\nu'}%\(
A_{\nu,\nu'} L_{\nu'} 
%\)
\end{eqnarray}
\end{rmk}

\begin{rmk}
When $X^{\sol}_{\nu'}$ are chosen such that $L_{\nu'}=0$ the term
$\sum_{\nu'} A_{\nu,\nu'} L_{\nu'}$
may vanish from the right hand side.
\end{rmk}

% The elimination of the local closure laws (Step 1) is achieved 
% for each control volume $\nu\in{\mathcal{M} \cup \mathcal{F}_\Gamma \cup \mathcal{V}} $ by splitting the unknowns $X_{Q_\nu}$ into $\#{\cal C}+1$ primary unknowns 
% $X_{Q_\nu}^{pr}$ and $N^{sd}_\nu$ secondary unknowns $X_{Q_\nu}^{sd}$ with 
% $$
% N^{sd}_\nu =   1 + \# Q_\nu + \sum_{\alpha\in Q_\nu} \#{\cal C}^\alpha 
% +\#\overline{\cal C}_{Q_\nu} -\#{\cal C}. 
% $$ 
% For each control volume $\nu\in{\mathcal{M} \cup \mathcal{F}_\Gamma \cup \mathcal{V}}  $, 
% the secondary unknowns are  chosen in such a way that the square matrix 
% $$
% {\partial L \over \partial X^{sd}_{Q_\nu} } \left(X^{pr}_{Q_\nu},X_{Q_\nu}^{sd}, Q_\nu\right)
% \in \mathbb{R}^{N^{sd}_\nu \times N^{sd}_\nu},  
% $$
% is non-singular. This choice can be done algebraically in the general case, 
% or defined once and for all for each set of present phases $Q_\nu$ 
% for specific physical cases. Here we remark that the unknowns $(n_{i,\nu})_{i\in\overline{\cal C}_{Q_\nu}}$ are not involved in the closure laws \eqref{VAG_closure} and hence are 
% always chosen as primary unknowns. 


 
\end{document}

