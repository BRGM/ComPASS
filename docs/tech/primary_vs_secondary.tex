
\documentclass{article}

\usepackage{ComPASS_notations}

\begin{document}

We are solving for the non linear system of equations $\mathcal{R} (X_\D) = 0$.
For each control volume $\nu \in \CV$, $\mathcal{R_\nu}$ is made of
conservation equations $R_\nu \(X_\D\)=0$ that involve adjacent degrees of
freedom and local closure laws $L_\nu \(X_\D\)=0$ that only depends on $X_\nu$,
local meaning that $\nabla_{X_\D}$ is block diagonal.

Let us write the equation in vector form:

%
\begin{eqnarray}
\label{NonLinearSystem}
\mathbf{0} = \mathcal{R} (X_\D) :=
\left\{\begin{array}{ll}
\left(\begin{array}{c}
R_\nu (X_\D) \\
L_\nu (X_\D)
\end{array}\right)
\ \nu \in \CV, \\
\end{array}\right.
\end{eqnarray}
%

Let us denote:
\begin{itemize}
\item $X^{\nat}=\(X^{\nat}_{\nu}\)$ the {\em physical / natural} unknowns
\item $X^{\sol}=\(X^{\sol}_{\nu}\)$ the unkwowns used to solve the problem.
\end{itemize}

such that: $X^{\sol} = \cG \(X^{\nat}\)$
and $X^{\sol}_{\nu} = \cG_\nu \( X^{\nat}_{\nu} \)$


We assume that $X^{\sol}_{\nu}$ writes in vector form :
$$
X^{\sol}_{\nu}=\(
\begin{array}{cc}
X^{\primu}_{\nu} \\
X^{\secdu}_{\nu}
\end{array}
\)
$$
and is such that the square matrix
$\nabla_{X^{\secdu}_{\nu}} L_\nu$ is non-singular
for all $\nu \in \CV$.

Most of the time $\cG$ is a mere permutation matrix.

When solving \eqref{NonLinearSystem} with respect to $X^{\sol}$ with a Newton
algorithm, we are seeking increments such that:

$$
\nabla_{X^{\sol}}\mathcal{R} \Delta_X = - \mathcal{R} \( X \)
$$

where the diagonal of the Jacobian $\nabla_{X^{\sol}}\mathcal{R}$ is made
of blocks of the form:

$$
\(\begin{array}{cc}
\nabla_{X^{\primu}_{\nu}} R_{\nu} &
\nabla_{X^{\secdu}_{\nu}} R_{\nu} \\
\nabla_{X^{\primu}_{\nu}} L_{\nu} &
\nabla_{X^{\secdu}_{\nu}} L_{\nu}
\end{array}\)
$$

As the rows involving $L^{\sol}_{\nu}$ are null except in this block (local
nature of the closure laws) we can eliminate these equations (which can be seen
as a Schur complement without fill-in):

Indeed:

$$
\nabla_{X^{\primu}_{\nu}} L_{\nu} \Delta_{X^{\primu}_{\nu}} +
\nabla_{X^{\secdu}_{\nu}} L_{\nu} \Delta_{X^{\secdu}_{\nu}}
= -L_{\nu}
$$

gives:

\begin{eqnarray}
\label{eq:scd_variations}
\Delta_{X^{\secdu}_{\nu}}
=  - \(\nabla_{X^{\secdu}_{\nu}} L_{\nu} \)^{-1} \(
L_{\nu} +
\nabla_{X^{\primu}_{\nu}} L_{\nu} \Delta_{X^{\primu}_{\nu}}
\)
\end{eqnarray}

At each step of the Newton algorithm we can choose
$( X^{\sol}_{\nu} )_{\nu\in\CV}$ that
$ L^{\sol}_{\nu} = 0 $. \footnote{This corresponds to a projection
onto the closure laws and can be implemented in the {\em flash}.}

Then for each degree of freedom $\nu'$ the rows involving $R^{\sol}_{\nu}$
rewrite:
$$
\sum_{\nu'}
\(
\nabla_{X^{\primu}_{\nu'}} R_{\nu} \Delta_{X^{\primu}_{\nu'}}+
\nabla_{X^{\secdu}_{\nu'}} R_{\nu} \Delta_{X^{\secdu}_{\nu'}}
\)
= R_{\nu}
$$

$$
\sum_{\nu'}
\[
\nabla_{X^{\primu}_{\nu'}} R_{\nu} \Delta_{X^{\primu}_{\nu'}}-
\nabla_{X^{\secdu}_{\nu'}} R_{\nu} \(
\(\nabla_{X^{\secdu}_{\nu'}} L_{\nu'} \)^{-1} \(
L_{\nu'} +
\nabla_{X^{\primu}_{\nu'}} L_{\nu'} \Delta_{X^{\primu}_{\nu'}}
\)
\)
\]
= R_{\nu}
$$

\begin{eqnarray}
\label{eq:newtonjacobian}
\left.\begin{array}{lr}
\sum_{\nu'}
\[
\nabla_{X^{\primu}_{\nu'}} R_{\nu}
- \nabla_{X^{\secdu}_{\nu'}} R_{\nu}
\( \nabla_{X^{\secdu}_{\nu'}} L_{\nu'} \)^{-1}
\nabla_{X^{\primu}_{\nu'}} L_{\nu'}
\] \Delta_{X^{\primu}_{\nu'}}
= \\
R_{\nu} + \sum_{\nu'} \[
\nabla_{X^{\secdu}_{\nu'}} R_{\nu}
\(\nabla_{X^{\secdu}_{\nu'}} L_{\nu'} \)^{-1} L_{\nu'}
\]
\end{array}\right.
\end{eqnarray}

We also have:
$$
\nabla_{X^{\sol}_{\nu}} F
= \nabla_{X^{\nat}_{\nu}} F \nabla_{X^{\sol}_{\nu}} X^{\nat}_{\nu}
$$
and:
$$
\nabla_{X^{\sol}_{\nu}} X^{\nat}_{\nu}
= \nabla_{X^{\sol}_{\nu}} \( \cG_{\nu}^{-1}\)
= \( \nabla_{X^{\nat}_{\nu}} \cG_{\nu} \) ^{-1}
$$

Let us note :
$$
\G_\nu := \nabla_{X^{\nat}_{\nu}} \cG_{\nu}
$$

And:
$$
\nabla_{X^{\sol}_{\nu}} F
= \nabla_{X^{\nat}_{\nu}} F {\G_{\nu}}^{-1}
$$

Introducing the notations
\footnote{note that when $\nu\neq\nu'$ we have $ \nabla_{X^{\nat}_{\nu'}} L_{\nu}=0$}:

$$
{\G_{\nu}}^{-1} = \[\,
\iGcol^{\primu}_{\nu} \, \iGcol^{\secdu}_{\nu}
\,\]
$$
$$
J^{R}_{\nu,\nu'} = \nabla_{X^{\nat}_{\nu'}} R_{\nu}
$$
$$
J^{L}_{\nu} = \nabla_{X^{\nat}_{\nu}} L_{\nu}
$$

we have:

$$
\nabla_{X^{\sol}_{\nu'}} R_{\nu}
=
\[\, \nabla_{X^{\primu}_{\nu'}} R_{\nu} \; \nabla_{X^{\secdu}_{\nu'}} R_{\nu} \,\]
 =
\[\, J^{R}_{\nu,\nu'} \iGcol^{\primu}_{\nu'} \; J^{R}_{\nu,\nu'} \iGcol^{\secdu}_{\nu'} \,\]
$$
$$
\nabla_{X^{\sol}_{\nu}} L_{\nu}
=
\[\, \nabla_{X^{\primu}_{\nu}} L_{\nu} \; \nabla_{X^{\secdu}_{\nu}} L_{\nu} \,\]
=
\[\, J^{L}_{\nu} \iGcol^{\primu}_{\nu} \; J^{L}_{\nu} \iGcol^{\secdu}_{\nu} \,\]
$$

so that \eqref{eq:newtonjacobian} rewrites :

\begin{eqnarray}
\label{eq:newtonjacobian2}
\left.\begin{array}{lr}
\sum_{\nu'}
\[
 J^{R}_{\nu,\nu'}
-  J^{R}_{\nu,\nu'} \iGcol^{\secdu}_{\nu'}
\(  J^{L}_{\nu'} \iGcol^{\secdu}_{\nu'} \)^{-1}
 J^{L}_{\nu'}
\] \iGcol^{\primu}_{\nu'} \Delta_{X^{\primu}_{\nu'}}
\\ =
R_{\nu} - \sum_{\nu'} \[
J^{R}_{\nu,\nu'} \iGcol^{\secdu}_{\nu'}
\( J^{L}_{\nu'} \iGcol^{\secdu}_{\nu'} \)^{-1}
L_{\nu'}
\]
\end{array}\right.
\end{eqnarray}

And finally \ref{eq:scd_variations} writes:

$$
\Delta_{X^{\secdu}_{\nu}}
=  - \( J^{L}_{\nu} \iGcol^{\secdu}_{\nu} \)^{-1} \(
L_{\nu} +
J^{L}_{\nu} \iGcol^{\primu}_{\nu} \Delta_{X^{\primu}_{\nu}}
\)
$$



\begin{rmk}
When $\cG$ is a permutation matrix the effect of  the $\iGcol^{\cdot}_{\nu}$
terms is just to extract columns/rows and is implemented manually in the code.
\end{rmk}

\begin{rmk}
The terms $A_{\nu,\nu'}=J^{R}_{\nu,\nu'} \iGcol^{\secdu}_{\nu'}
\( J^{L}_{\nu'} \iGcol^{\secdu}_{\nu'} \)^{-1}$ may be stored as they
appear twice, so that \eqref{eq:newtonjacobian2} would become:
\begin{eqnarray}
\sum_{\nu'}
\[
 J^{R}_{\nu,\nu'}
-  A_{\nu,\nu'}
 J^{L}_{\nu',\nu'}
\] \iGcol^{\primu}_{\nu'} \Delta_{X^{\primu}_{\nu'}}
=
R_{\nu} - \sum_{\nu'}%\(
A_{\nu,\nu'} L_{\nu'}
%\)
\end{eqnarray}
\end{rmk}

\begin{rmk}
When $X^{\sol}_{\nu'}$ are chosen such that $L_{\nu'}=0$ the term
$\sum_{\nu'} A_{\nu,\nu'} L_{\nu'}$
may vanish from the right hand side.
\end{rmk}

%%%%%%%%%%%%%%%%%%%%%%%%%%%%%%%%%%%%%%%%

\subsection*{Special case}

Let's suppose that: $\cG$ is such that:

$$\cG: X^{\nat} = \[X^{\nat}_k\]_{1\leq k \leq n}
\mapsto X^{\sol} = \cG \(X^{\nat}\) = \[X^{\nat}_{\tau\(k\)}\]_{1\leq k \leq n} $$

where $\tau$ is a permutation over $\llbracket 1, n \rrbracket$.
Then we have trivially:
$$\G=\nabla_{X^{\nat}} \cG=\[ \delta_{\tau\(i\),j}\]_{1\leq i,j \leq n}$$

and:
\begin{eqnarray}
\label{eq:invG}
\G^{-1}=\[ \delta_{\tau^{-1}\(i\),j}\]_{1\leq i,j \leq n}
\end{eqnarray}

Let us note $\ns$ the number of primary variables and  $\ns$ the number of secondary
variables such that we have $\np + \ns = n$. From \ref{eq:invG} it comes:

$$\iGcol^{\primu}=\[ \delta_{\tau^{-1}\(i\),j}\]_{1\leq j \leq \np}$$
$$\iGcol^{\secdu}=\[ \delta_{\tau^{-1}\(i\),n_p+j}\]_{1\leq j \leq \ns}$$

Considering the matrix $A$, the effect of the multiplication $A\iGcol^{\primu}$
({\em resp.} $A\iGcol^{\secdu}$)
is just to extract the columns of A corresponding to the primary variables
({\em resp.} the secondary variables)\footnote{
Considering $A=\[ a_{i,j}\]$, we have:
$$
A \iGcol^{\primu} = \[
 \sum_{1\leq k \leq n} a_{i,k} \delta_{\tau^{-1}\(k\),j}
 \]_{\(i, j\) \in \llbracket 1, n \rrbracket \times \llbracket 1, \np \rrbracket}
$$
$$
A \iGcol^{\primu} = \[
 a_{i,\tau\(j\)}
 \]_{\(i, j\) \in \llbracket 1, n \rrbracket \times \llbracket 1, \np \rrbracket}
$$}.

Considering $\mathcal{X}$ the set of all the Coats {\em natural variables}
$\mathcal{X}^{\nat}_\ctx$ may be a subset of  $\mathcal{X}$ that depends on the
 so-called {\em physical context $\ctx$}. Whereas $\np$ is equal to the number of balance
equations invloving fluxes between degrees of freedom, $\ns$ may vary, depending
on the context $\ctx$. Variables that belong to $\mathcal{X} \setminus \mathcal{X}^{\nat}_\ctx$ have a constant
value in the context $\ctx$, so that their value is set upon entering the context.

\begin{rmk}
Considering that whe select $\mathcal{X}^{\nat}_\ctx$ in $\mathcal{X}$ we can
always make this selection such that $\tau$ is the identity. In this case
derivatives must be carefully indexed.
\end{rmk}

%%%%%%%%%%%%%%%%%%%%%%%%%%%%%%%%%%%%%%%%%%%

\subsection*{One component two phases}

We have two monophasic contexts and one diphasic context.

Molar fractions are always constant (a single component) with $C^\alpha=1$

\begin{itemize}

\item In monophasic contexts $\mathcal{X}^{\nat}_\ctx = \{ p, T\}$ and $\np=2$, $\ns=0$.
Saturations are constant with $S^\alpha\in\{0,1\}$ depending on the context.

\item In diphasic context $\mathcal{X}^{\nat}_{diphasic} = \{ p, T, S^g, S^l\}$ and $\np=\ns=2$ with
 $\mathcal{X}^{\primu}_{diphasic} = \{ p, S^g\}$ and $\mathcal{X}^{\secdu}_{diphasic} = \{ T, S^l\}$.
We have :
$$
L = \left\{
\begin{array}{l}
T - T_{sat}\(p\) = 0 \\
S^g + S^l - 1 = 0 \\
\end{array}
\right.
$$

So that:

$$
 J^{L}_{\nu} = \[
\begin{array}{cccc}
-\frac{\partial T_{sat}}{\partial p} & 1 & 0 & 0 \\
0 & 0 & 1 & 1 \\
\end{array}
\]
$$

and:

$$
\nabla_{X^{\secdu}} L
=
J^{L} \iGcol^{\secdu}
= \[
\begin{array}{cccc}
1 & 0 \\
0 & 1 \\
\end{array}
\]
= I
$$


\end{itemize}

% The elimination of the local closure laws (Step 1) is achieved
% for each control volume $\nu\in{\mathcal{M} \cup \mathcal{F}_\Gamma \cup \mathcal{V}} $ by splitting the unknowns $X_{Q_\nu}$ into $\#{\cal C}+1$ primary unknowns
% $X_{Q_\nu}^{pr}$ and $N^{sd}_\nu$ secondary unknowns $X_{Q_\nu}^{sd}$ with
% $$
% N^{sd}_\nu =   1 + \# Q_\nu + \sum_{\alpha\in Q_\nu} \#{\cal C}^\alpha
% +\#\overline{\cal C}_{Q_\nu} -\#{\cal C}.
% $$
% For each control volume $\nu\in{\mathcal{M} \cup \mathcal{F}_\Gamma \cup \mathcal{V}}  $,
% the secondary unknowns are  chosen in such a way that the square matrix
% $$
% {\partial L \over \partial X^{sd}_{Q_\nu} } \left(X^{pr}_{Q_\nu},X_{Q_\nu}^{sd}, Q_\nu\right)
% \in \mathbb{R}^{N^{sd}_\nu \times N^{sd}_\nu},
% $$
% is non-singular. This choice can be done algebraically in the general case,
% or defined once and for all for each set of present phases $Q_\nu$
% for specific physical cases. Here we remark that the unknowns $(n_{i,\nu})_{i\in\overline{\cal C}_{Q_\nu}}$ are not involved in the closure laws \eqref{VAG_closure} and hence are
% always chosen as primary unknowns.

\end{document}
